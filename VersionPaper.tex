\documentclass[conference]{IEEEtran}

\hyphenation{op-tical net-works semi-conduc-tor}

% !TEX root =  ./VersionPaper.tex
\usepackage[english]{babel}
\usepackage{threeparttable}
\usepackage{amssymb}
\setcounter{tocdepth}{3}
\usepackage{graphicx}
\usepackage{multirow}
\usepackage{url}
\usepackage{array}
\usepackage{xspace}
\usepackage{moreverb}
\usepackage{listings}
\usepackage{color}
\usepackage{cite}
\usepackage{relsize}
\usepackage{color}
%\usepackage{wasysym}
\usepackage{fancybox}
\usepackage{balance}
\usepackage{enumitem}
\usepackage{lmodern}
%\usepackage{caption}
\usepackage{epstopdf}

\usepackage{amsmath}

%\usepackage[font={bf}, labelfont=bf]{caption}
%\usepackage[font={bf,small},labelfont=bf]{caption}
\usepackage{datetime}
%\captionsetup[table]{skip=5pt}

\newcommand{\version}[1]{\normalsize{Version #1 - \mydate\today\xspace,\xspace\currenttime}}

\newdateformat{mydate}{\monthname[\THEMONTH]\xspace\THEDAY}

\newcommand{\fixme}[1]{\textcolor{red}{#1}}
\newcommand{\TODO}[1]{\textcolor{magenta}{\textbf{TODO: }#1}}
\newcommand{\feedback}[1]{\textcolor{magenta}{\textbf{ASK: }#1}}



%\newcommand{\MyDefine}[1]{\textit{\textbf{#1}}

\newcommand{\ADD}[1]{\textbf{\textcolor{blue}{#1}}}
\newcommand{\REMOVE}[1]{\textbf{\textcolor{red}{#1}}}

\newcommand{\mybox}[1]{
\begin{center}
\setlength{\fboxsep}{5pt}%
\Ovalbox{%
\begin{minipage}{.45\textwidth}
\begin{center}
{\it #1} 
\end{center}
\end{minipage}}
\end{center}
}


\lstdefinelanguage{CSharp}
{
sensitive=true,
morekeywords=[1]{
abstract, as, async, await, base, break, case,
catch, checked, class, const, continue,
default, delegate, do, else, enum,
event, explicit, extern, false,
finally, fixed, for, foreach, goto, if,
implicit, in, interface, internal, is,
lock, namespace, new, null, operator,
out, override, params, private,
protected, public, readonly, ref,
return, sealed, sizeof, stackalloc,
static, struct, switch, this, throw,
true, try, typeof, unchecked, unsafe,
using, virtual, volatile, while, bool,
byte, char, decimal, double, float,
int, lock, object, sbyte, short, string,
uint, ulong, ushort, void},
morecomment=[l]{//},
morecomment=[s]{/*}{*/},
morecomment=[l][keywordstyle4]{\#},
morestring=[b]",
morestring=[b]',
}

%\usepackage[labelfont=bf]{caption}
%\DeclareCaptionFormat{listing}{\rule{\dimexpr\textwidth+17pt\relax}{0.4pt}\par\vskip1pt#1#2#3}
%\captionsetup[lstlisting]{singlelinecheck=false, margin=0pt, font={sf},labelsep=space,labelfont=bf}
%\renewcommand\lstlistingname{Code}

\definecolor{mygreen}{rgb}{0,0.6,0}
\definecolor{mygray}{rgb}{0.5,0.5,0.5}
\definecolor{mymauve}{rgb}{0.58,0,0.82}

\lstset{ %
%frame=bottom,
  language=CSharp,                % the language of the code
  basicstyle=\ttfamily\scriptsize,           % the size of the fonts that are used for the code
  keywordstyle=\color{blue},
  commentstyle=\color{mygreen},
  numberstyle=\tiny\color{mygray}
  stringstyle=\color{mymauve},
  rulecolor=\color{black}, 
  stepnumber=1,                   % the step between two line-numbers. If it's 1, each line 
  %numbers=left,                   % where to put the line-numbers
  numberstyle=\scriptsize,  
  numbersep=5pt,  
  showspaces=false,               % show spaces adding particular underscores
  showstringspaces=false,         % underline spaces within strings
  showtabs=false,                 % show tabs within strings adding particular underscores
  tabsize=2,                      % sets default tabsize to 2 spaces
  breaklines=false,                % sets automatic line breaking
  breakatwhitespace=false,        % sets if automatic breaks should only happen at whitespace
  belowskip=1pt,
  aboveskip=2pt
}


\newcommand{\codesnippet}[1]{
\begin{lstlisting}#1\end{lstlisting}}

\newcommand{\readme}[1]{#1}

\newcommand{\etal}{\emph{et~al}.~}
\newcommand{\Comment}[1]{}%\textbf{\textsl{$\langle\!\langle$#1$\rangle\!\rangle$}}}
\newcommand{\spacedinlineheader}[1]{\vspace{1.5 mm} \noindent \textbf{#1}}
\newcommand{\Space}[1]{}


\newcommand{\TotalNumProjects}{41\xspace}
\newcommand{\numberchoices}{\textit{number of choices for assigning a new version number}\xspace}
\newcommand{\numberextrachoices}{\textit{number of additional choices for assigning a new version number}\xspace}
\newcommand{\choices}{\textit{choices for assigning a new version number}\xspace}
\newcommand{\choice}{\textit{choice for assigning a new version number}\xspace}

\newcommand{\code}[1]{\begin{small}\texttt{#1}\end{small}}
\newcommand{\URL}[1]{\begin{footnotesize}\url{#1}\end{footnotesize}}

\newcommand{\Analyzer}{\begin{small}\textsc{VersionAnalyzer}\end{small}\xspace}


\newcommand{\RQOne}{\textbf{RQ1:} How do OSS vs. proprietary development affect version numbers?\xspace}

\newcommand{\subParagraph}[1]{\textbf{\emph{#1}}}
\newcommand{\MyParagraph}[1]{\noindent \textbf{#1}}

\begin{document}

\title{A study on version number practices}

\author{\IEEEauthorblockN{Sergii Shmarkatiuk, 
Kendall Bailey and
Danny Dig
}}

%\numberofauthors{1} %  in this sample file, there are a *total*
%% of EIGHT authors. SIX appear on the 'first-page' (for formatting
%% reasons) and the remaining two appear in the \additionalauthors section.
%%
%\author{
%% 1st. author
%\alignauthor
%Sergii Shmarkatiuk, Kendall Bailey, Danny Dig\\
%       \affaddr{Oregon State University}\\
%       \email{\{shmarkas,baileken,digd\}@eecs.oregonstate.edu}
%}

\maketitle
\begin{abstract}
 TODO Abstract
 TODO Category
 \end{abstract}


\IEEEpeerreviewmaketitle

\section{Introduction}
TODO Last



\section{Experimental Setup}

\subsection{Data Corpus}

\par To answer our research questions we gathered information about version numbering from both open source and proprietary projects, totaling  \TotalNumProjects projects. When selecting projects to add to the corpus we chose
projects that are (i) mature and well-known, and (ii) representative. Our corpus spans the areas of software development tools, operating systems, games, programming languages, desktop applications, productivity tools, libraries and frameworks, etc. 
Our requirements for members of the corpus also included easily accessible version information as well as adequate documentation.

For open source projects we drew data from both Github and SourceForge because previous studies \cite{} have shown that git and svn, the backbone of Github and SourceForge respectively, are the most widely used version control systems. In addition to this, both systems contain rich history information about tags and version numbers. To retrieve the version history of each project we used the \code{git tag} and \code{svn list} 
%svn://svn.code.sf.net/p/\$project\_name/code/tags}  
commands and stored the output of each command for later analysis.

For proprietary projects, sometimes the version numbers from the producer website
were incomplete or disorganized. Thus we recovered the missing information from third party websites.
To extract version numbers, we combined information from readily available producer sources such as release notes, documentation, and download archives
with information from third party websites such as wikis, user forums, press releases.
When we had to use multiple sources, we merged the version numbers into a single set per project. During the merge we removed duplicates between the documents, but did not remove duplicates when they existed in the same document, e.g., when two different download archives had the same version number.

Table \ref{tab:ProjectsCorpus} lists the names of the projects in our corpus. For each project we also tabulate information such as age, size, number of versions, \TODO{describe the columns}.

\subsection{Data Parsing}

 Version numbering patterns can vary widely between projects, thus making comparison difficult. 
To counter this, we defined common patterns across different numbering approaches. 

We defined a version number as a composition of the following parts: 
prefix, first version compound, second
 version compound, third version compound, fourth version compound, suffix label, and suffix number. 

Additionally, our definition discards delimiters (e.g., period, underscore, space, dash). When the version number contains textual compounds (e.g., \code{a}, \code{b}, \code{c}) we convert them into a numerical representation such that we can compute metrics even for such compounds. 

Table \ref{table:exampleVCs} shows examples of version numbers before-and-after 
our parsing.
 
 \begin{table*}[t]
%\resizebox{1.4\textwidth}{!}{\begin{minipage}{\textwidth}
\begin{center}
\begin{tabular}{| *{8}{l|}}
\hline
Version Number & Prefix & 1 VC & 2 VC & 3 VC & 4 VC  & Suffix Label & Suffix Number \\ \hline
v\_3.4.5\_p34 & v\_ & 3 & 4 & 5 & & p & 34 \\
R 2.11.1 RC & R & 2 & 11 & 1 & & RC & \\
5.6.32.78 &  & 5 & 6 & 32 & 78 & & \\ \hline

\end{tabular}
\end{center}
\caption{Numeric version compounds are bolded. }
\label{table:exampleVCs}
%\end{minipage} }
\end{table*}
 
 
 
 To process and analyze the collected version number data we wrote a script that parses the data into our format. 


%TODO link to github!
%TODO Editting stopped here

% In order to ensure that we covered most common patterns of version numbering that appear in the dataset, we created a test suite with common parsing scenarios (TODO: do we need to list common scenarios? ). Test cases can be found in github repository [LINK] of our project.  



% In order to ensure that we covered most common patterns of version numbering that appear in the dataset, we created a test suite with common parsing scenarios (TODO: do we need to list common scenarios? ). Test cases can be found in github repository [LINK] of our project.  


%\par Different projects use different version numbering practices. 
%In order to standardize version numbering patterns to be able to parse corresponding versions that appear in different data sources, we have come up with a convention for version number structure. 
%Convention includes following parts of version number: prefix, first version compound, second version compound, third version compound, fourth version compound, suffix label, suffix number. 
%Script that parses version numbers allows for following separators between different version number compounds: '.' (period), '-' (dash), '\_'(underscore), ' ' (space). 
%You can find examples of version number parsing results in Table \ref{table:exampleVCs}. Also we distinct between numeric version compounds and textual version compound. Numeric version compounds appear in Table \ref{table:exampleVCs} in bold.




\section{Metrics}


In order to analyze the degree of inconsistency in version numbers, we define 
a set of metrics that measure common features of version numbers sequences: 
 
\TODO{Use the Latex command for definition:} A jump is defined by two consecutively released versions when a version compound increases by more than one.
For example, one project where version 1.0.25 is followed immediately by version 1.0.28 represents one jump in the third version compound. This jump's size is 3.

Jumps represent inconsistencies in version numbering and can cause confusion.
 
%\begin{itemize}
\MyParagraph{\textbf{Number of jumps}}.  For each version compound of a project, this metric counts the total number of jumps across all versions. For example, in a project where one jump happens from version 1.0.25 to 1.0.28, and another jump happens from 1.0.34 to 1.0.40 will register as two jumps in the third version compound.\\

\MyParagraph{\textbf{Sum of jumps}}.  For each version compound of a project, this metric counts the sum of all sizes of jumps within that compound. 
In the example above, the length of jump on the third version compound is 
9 ( 3 + 6). Whereas the previous metric records whether jumps occurred, this metric records the severity of the jumps. \\
 
\MyParagraph{\textbf{Average size of jumps}}. For each version compound of a project, this metric calculates average of  jump sizes within that compound. For the example above, the average jump size for the third version compound is 4.5 ( ( 3 + 6 ) / 2)

%\begin{equation}
%VC_{2} - VC_{1} > 1
%\end{equation}

%\end{itemize}

We collect length of each jump for further analysis. For example list of jumps for third version compound for array of version numbers [1.0.2, 1.0.3, 1.0.5, 1.0.6, 1.0.7, 1.0.10] would be [2, 3]. 

\begin{itemize}

\item \textbf{Number of empty jumps} for each numeric version compound. We consider change of numeric version compound to be an empty jump if empty value substitutes numeric value of version compound. For example, there is an empty jump of fourth version compound for the pair of version numbers [1.0.2.4, 1.0.3].


\begin{equation}
\begin{split}
isNumeric(VC_{1}) \&\& isEmpty(VC_{2})  || \\  
isNumeric(VC_{2}) \&\& isEmpty(VC_{1})
\end{split}
\end{equation}

 \textbf{Number of version placeholders} for each numeric version compound. Sometimes version numbering patterns use special characters to represent set of version numbers instead of just one version number. For example, version number 1.x represents set of version numbers [1.0, 1.1, 1.2, $\dots$, 1.N]. As long as we found only one special character that some software projects use as a version placeholder ('x'), we count only for appearances of this character as version compound value. 
 
 \textbf{Instances of Megalomania} for list of versions. 
There are several ways of incrementing version numbers in cases when version numbers consist of more than one version compound. For example developer might think about two ways of incrementing version 1.0 if he or she needs to come up with version for next release. As far as most of the existing conventions describe [LINKS], first way is to assign version 1.1 to new release. Second way is to assign version 2.0. Note that second approach uses increment of first version compound instead of incrementing rightmost version compound. 

We consider this case of assigning next version number (incrementing version compounds on the left side instead of rightmost version compound) to be instance of megalomania. 

In most general case, instances of megalomania include increment of version number than involves increment of other version compounds than the rightmost version compound.

Other examples of megalomania: $$ 1.0.2 \rightarrow 2.0.0 \texttt{, } 1.4.5 \rightarrow 1.5.0 $$

There are N-1 possible cases of assigning new version to the version number with N numeric version compounds. We distinct between different ways of assigning version numbers depending on position of incremented numeric version compound. 

For example we consider change of version 1.0.0.0 to 2.0.0.0 to be megalomania of severity 3, change of version 1.0.4\_p34 to 1.1.0 to be megalomania of severity 2, etc. In most general case, we consider changing 1st numeric version compound in version with N numeric version compounds to be megalomania of severity N-1, 2nd numeric version compound in version with N numeric version compounds to be megalomania of severity N-2, and so on. 

We collect information about megalomania severity as another metric. For example, list of megalomania severities for list of versions [1.0.0, 1.0.1, 1.0.2, 1.1.0, 1.1.1, 1.2.0, 1.2.1, 2.0.0] would be [1,1,2].

We calculated metrics both for each version compound and aggregated values to represent metrics for  version number.
\end{itemize}

\TODO{Make the next paragraph flow with the rest}
Our script collects metrics by traversing list of version numbers pair by pair. Single value for each metric takes a pair of version numbers [$VN_{1}$, $VN_{2}$] to produce metric value.

\section{Analysis}

\begin{itemize}
\item RQ1: OSS vs Proprietary
\item RQ2: Projects of different sizes
\item RQ3: Age of different projects
\item RQ4: Dev tools vs other applications
\item RQ5: SDKs vs other applications 
\end{itemize}


\section{Implications}

\section{Threats to Validity}

 The proprietary data may be incomplete because we only harvested information that the distributor chose to display online which could contribute to additional jumps for artifacts that were never made public. Conversely, enterprises may shield the general from the burden of intermediate builds by only displaying clean, sequential versions on their websites and in their documentation which could skew the results of our analysis.





\section{Conclusions}


\section*{Acknowledgments}

The authors would like to thank...

%
% The following two commands are all you need in the
% initial runs of your .tex file to
% produce the bibliography for the citations in your paper.
\bibliographystyle{IEEEtran}
\bibliography{bibliography}  % sigproc.bib is the name of the Bibliography in this case

\end{document}
